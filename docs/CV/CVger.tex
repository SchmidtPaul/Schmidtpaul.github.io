%!TEX TS-program = xelatex
%!TEX encoding = UTF-8 Unicode
% Awesome CV LaTeX Template for CV/Resume
%
% This template has been downloaded from:
% https://github.com/posquit0/Awesome-CV
%
% Author:
% Claud D. Park <posquit0.bj@gmail.com>
% http://www.posquit0.com
%
%
% Adapted to be an Rmarkdown template by Mitchell O'Hara-Wild
% 23 November 2018
%
% Template license:
% CC BY-SA 4.0 (https://creativecommons.org/licenses/by-sa/4.0/)
%
%-------------------------------------------------------------------------------
% CONFIGURATIONS
%-------------------------------------------------------------------------------
% A4 paper size by default, use 'letterpaper' for US letter
\documentclass[11pt,a4paper,]{awesome-cv}

% Configure page margins with geometry
\usepackage{geometry}
\geometry{left=1.4cm, top=.8cm, right=1.4cm, bottom=1.8cm, footskip=.5cm}


% Specify the location of the included fonts
\fontdir[fonts/]

% Color for highlights
% Awesome Colors: awesome-emerald, awesome-skyblue, awesome-red, awesome-pink, awesome-orange
%                 awesome-nephritis, awesome-concrete, awesome-darknight

\definecolor{awesome}{HTML}{414141}

% Colors for text
% Uncomment if you would like to specify your own color
% \definecolor{darktext}{HTML}{414141}
% \definecolor{text}{HTML}{333333}
% \definecolor{graytext}{HTML}{5D5D5D}
% \definecolor{lighttext}{HTML}{999999}

% Set false if you don't want to highlight section with awesome color
\setbool{acvSectionColorHighlight}{true}

% If you would like to change the social information separator from a pipe (|) to something else
\renewcommand{\acvHeaderSocialSep}{\quad\textbar\quad}

\def\endfirstpage{\newpage}

%-------------------------------------------------------------------------------
%	PERSONAL INFORMATION
%	Comment any of the lines below if they are not required
%-------------------------------------------------------------------------------
% Available options: circle|rectangle,edge/noedge,left/right

\photo{paulzoomcircle.png}
\name{Dr.}{Paul Schmidt}

\position{Data Scientist / Biostatistiker}
\address{Hamburg, Deutschland}

\mobile{+49 172 3091577}
\email{\href{mailto:schmidtpaul1989@outlook.com}{\nolinkurl{schmidtpaul1989@outlook.com}}}
\researchgate{Paul\_Schmidt17}
\github{SchmidtPaul}
\linkedin{schmidtpaul1989}

% \gitlab{gitlab-id}
% \stackoverflow{SO-id}{SO-name}
% \skype{skype-id}
% \reddit{reddit-id}


\usepackage{booktabs}

\providecommand{\tightlist}{%
	\setlength{\itemsep}{0pt}\setlength{\parskip}{0pt}}

%------------------------------------------------------------------------------



% Pandoc CSL macros
\newlength{\cslhangindent}
\setlength{\cslhangindent}{1.5em}
\newlength{\csllabelwidth}
\setlength{\csllabelwidth}{2em}
\newenvironment{CSLReferences}[3] % #1 hanging-ident, #2 entry spacing
 {% don't indent paragraphs
  \setlength{\parindent}{0pt}
  % turn on hanging indent if param 1 is 1
  \ifodd #1 \everypar{\setlength{\hangindent}{\cslhangindent}}\ignorespaces\fi
  % set entry spacing
  \ifnum #2 > 0
  \setlength{\parskip}{#2\baselineskip}
  \fi
 }%
 {}
\usepackage{calc}
\newcommand{\CSLBlock}[1]{#1\hfill\break}
\newcommand{\CSLLeftMargin}[1]{\parbox[t]{\csllabelwidth}{\honortitlestyle{#1}}}
\newcommand{\CSLRightInline}[1]{\parbox[t]{\linewidth - \csllabelwidth}{\honordatestyle{#1}}}
\newcommand{\CSLIndent}[1]{\hspace{\cslhangindent}#1}

\begin{document}

% Print the header with above personal informations
% Give optional argument to change alignment(C: center, L: left, R: right)
\makecvheader

% Print the footer with 3 arguments(<left>, <center>, <right>)
% Leave any of these blank if they are not needed
% 2019-02-14 Chris Umphlett - add flexibility to the document name in footer, rather than have it be static Curriculum Vitae


%-------------------------------------------------------------------------------
%	CV/RESUME CONTENT
%	Each section is imported separately, open each file in turn to modify content
%------------------------------------------------------------------------------



\makecvfooter{Februar, 2023}{Dr. Paul Schmidt~~~·~~~Curriculum Vitae}{\thepage}

\hypertarget{berufserfahrung}{%
\section{Berufserfahrung}\label{berufserfahrung}}

\begin{cventries}
    \cventry{Data scientist / Geschäftsführer}{BioMath - Applied Statistics and Informatics in Life Sciences}{Rostock \& Hamburg}{Seit Jan 2019}{\begin{cvitems}
\item Verschiedene statistische Analysen von Rohdaten bis zum Schlussbericht für z.B. jährliches post-market Monitoring (Umfrage; Landwirtschaft), Risikobewertung (Metaanalyse; Epidemiologie), mehrjähriger Feldversuche (Experiment; Umwelt), Geografische Verteilung (GIS; Landesamt)
\item Implementierung neuer / Optimierung vorhandener SOPs (z.B. für systematic literature reviews und Metaanalysen), indem beispielsweise die Funktionalität vorhandener Software besser genutzt wird und zusätzlich ergänzende Software/Tools eingesetzt werden
\item Koordination der Kommunikation und des Zeitmanagements von Projekten
\item Durchführung von detaillierten Recherchen und Verfassen von wissenschaftlichen Texten
\item Geschäftsführer seit September 2022
\end{cvitems}}
    \cventry{Workshop Leiter}{Freelancer (nebenberuflich)}{siehe `Workshops' Abschnitt unten}{Seit Nov 2018}{\begin{cvitems}
\item Durchführung von Workshops zu Statistik mit R; der genaue Inhalt und die Kurssprache in Absprache mit dem Auftraggeber
\item Bereitstellung des Kursmaterials auf Webseite https://schmidtpaul.github.io/dsfair\_quarto/
\end{cvitems}}
    \cventry{Wiss. Mitarbeiter}{Universität Hohenheim}{Stuttgart}{Sep 2015 - Dez 2018}{\begin{cvitems}
\item Persönliche Beratung (von Einzeltermin bis projektbegleitend) für Studenten und wissenschaftliche Mitarbeiter hinsichtlich Versuchsdesign, Datenverarbeitung, statistischer Analysen und/oder Ergebnisdarstellung
\item Entwicklung, Organisation und Durchführung jährlicher statistischer Auswertungen von Versuchen zur Ertragsstabilität für eine externe Firma
\item Entwicklung, Organisation und Durchführung von Workshops zu Statistik mit R und SAS
\item Betreuung einer MSc Thesis
\end{cvitems}}
    \cventry{Junior Data scientist}{BioMath - Applied Statistics and Informatics in Life Sciences}{Rostock}{Jan 2015 - Aug 2015}{\begin{cvitems}
\item Optimierung statistischer Analysen von Monitoring-Daten
\item Implementierung von SOPs zu Systematic Literature Reviews
\end{cvitems}}
\end{cventries}

\hypertarget{ausbildung}{%
\section{Ausbildung}\label{ausbildung}}

\begin{cventries}
    \cventry{Dr. sc. agr.}{Universität Hohenheim}{Stuttgart}{Sep 2015 - Nov 2019}{\begin{cvitems}
\item DFG-geförderter Doktorand im Fachgebiet Biostatistik unter Prof. Dr. Hans-Peter Piepho
\item Kumulative Doktorarbeit: 'Estimating heritability in plant breeding programs' benotet mit 'magna cum laude'
\end{cvitems}}
    \cventry{Gast Doktorand}{Purdue University}{West Lafayette, IN, USA}{Sep 2015 - Dez 2015}{\begin{cvitems}
\item Gastdoktorand im Fachgebiet statistical bioinformatics unter Prof. Dr. Rebecca Whitbeck Doerge
\item Durch Eigeninitiative organisiert um den wissenschaftlichen Austausch und so die Inspiration zu Beginn meiner Doktorarbeit anzuregen
\end{cvitems}}
    \cventry{MSc Crop Science: Plant Breeding}{Universität Hohenheim}{Stuttgart}{Okt 2012 - Dez 2014}{\begin{cvitems}
\item Vertiefung in Biostatistik und Pflanzenzüchtung (Gesamtnote 1,4)
\item MSc Thesis: 'Statistical Evaluation and Analysis of PACTS trials as a series of on-farm strip trials without replicates' benotet mit 1,0
\end{cvitems}}
    \cventry{BSc Agrarbiologie}{Universität Hohenheim}{Stuttgart}{Okt 2009 - Sep 2012}{\begin{cvitems}
\item Vertiefung in Genetik und Pflanzenwissenschaften  (Gesamtnote 1,9)
\item BSc Thesis: 'Cumulative effects of glyphosate trace concentrations during root exposition of winter wheat' benotet mit 1,0
\end{cvitems}}
    \cventry{Schüleraustausch}{Alexander Central High School}{Taylorsville, NC, USA}{Aug 2006 - Jul 2007}{\begin{cvitems}
\item Vollendung des Abschlussjahres samt Erhalt eines High School Diploms
\end{cvitems}}
\end{cventries}

\hypertarget{fuxe4higkeiten}{%
\section{Fähigkeiten}\label{fuxe4higkeiten}}

\begin{cvskills} 
\cvskill {Generell} 
{Teamfähigkeit, Kommunikation, strukturiertes Arbeiten, Zeitmanagement, Problemlösung, zielorientiert } 

\cvskill {Open Source} 
{Webseite schmidtpaul.github.io/dsfair\_quarto/, R Paket BioMathR https://schmidtpaul.github.io/BioMathR/, R Paket CitaviR schmidtpaul.github.io/CitaviR/ } 

\cvskill {Präsentation} 
{Datenvisualisierung, Datenanalysebericht, wissenschaftliche Publikationen, Präsentationen } 

\cvskill {Software} 
{R, Python, SAS, SPSS, MS Office, Adoce Acrobat Pro, Latex, C\#, SQL } 

\cvskill {Sprachen} 
{Deutsch (Muttersprache), Englisch (kompetente, professionelle Sprachverwendung) } 

\cvskill {Statistik} 
{(generalisierte) lineare (gemischte) Modelle, explorative \& deskriptive Datenauswertung, Versuchsdesign } 
\end{cvskills}

\hypertarget{workshops}{%
\section{Workshops}\label{workshops}}

\begin{cvhonors} 
\cvhonor
{Statistics with R - an Introduction  }
{Universität Bonn via zoom}
{12h}
{Jul 2023  }

\cvhonor
{Statistics with R - an Introduction  }
{Universität Bonn via zoom}
{12h}
{May 2023  }

\cvhonor
{Introduction to data science for exp. life sciences with R  }
{Pro-RUWA via zoom}
{24h}
{Feb 2023  }

\cvhonor
{Data science for exp. life sciences with R (part 2)  }
{Thünen-Institut, Braunschweig via zoom}
{20h}
{Nov 2022  }

\cvhonor
{Data Science in den exp. Naturwiss. mit R (Teil 2)  }
{Thünen-Institut, Braunschweig via zoom}
{20h}
{Nov 2022  }

\cvhonor
{Data science for exp. life sciences with R (part 1)  }
{Thünen-Institut, Braunschweig via zoom}
{20h}
{Nov 2022  }

\cvhonor
{Data Science in den exp. Naturwiss. mit R (Teil 1)  }
{Thünen-Institut, Braunschweig via zoom}
{20h}
{Nov 2022  }

\cvhonor
{Statistics with R - an Introduction  }
{Universität Bonn via zoom}
{12h}
{Nov 2022  }

\cvhonor
{R and the {Tidyverse}  }
{FBN, Dummerstorf via zoom}
{5h}
{Oct 2022  }

\cvhonor
{Data science for exp. life sciences with R (part 2)  }
{Thünen-Institut, Braunschweig via zoom}
{24h}
{Mar 2022  }

\cvhonor
{Data Science in den exp. Naturwiss. mit R (Teil 2)  }
{Thünen-Institut, Braunschweig via zoom}
{24h}
{Mar 2022  }

\cvhonor
{Data science for exp. life sciences with R (part 1)  }
{Thünen-Institut, Braunschweig via zoom}
{24h}
{Mar 2022  }

\cvhonor
{Data Science in den exp. Naturwiss. mit R (Teil 1)  }
{Thünen-Institut, Braunschweig via zoom}
{24h}
{Mar 2022  }

\cvhonor
{Statistics with R (Beginner)  }
{Universität Kassel}
{24h}
{Dec 2021  }

\cvhonor
{Data science in den Naturwiss. mit R (Teil 2)  }
{Thünen-Institut, Braunschweig via zoom}
{24h}
{Jul 2021  }

\cvhonor
{Data science in den Naturwiss. mit R (Teil 1)  }
{Thünen-Institut, Braunschweig via zoom}
{24h}
{May 2021  }

\cvhonor
{Data science in den Naturwiss. mit R (Teil 2)  }
{Thünen-Institut, Braunschweig via zoom}
{24h}
{Mar 2021  }

\cvhonor
{Planning exp. designs, repeated measurements, and their analyses in R  }
{Universität Kassel via zoom}
{16h}
{Nov 2020  }

\cvhonor
{Data science in den Naturwiss. mit R (Teil 1)  }
{Thünen-Institut, Braunschweig via zoom}
{24h}
{Nov 2020  }

\cvhonor
{Experimental Design - Practicals in R  }
{CIHEAM Zaragoza via zoom}
{10h}
{Oct 2020  }

\cvhonor
{Real‑time consultation on statistics and mixed models in R  }
{Universität Kassel}
{16h}
{Mar 2020  }

\cvhonor
{Basics of applied statistics  }
{Universität Rostock}
{16h}
{Dec 2019  }

\cvhonor
{Data science for life sciences with R (part 2)  }
{Thünen-Institut, Braunschweig}
{24h}
{Nov 2019  }

\cvhonor
{Data science for life sciences with R (part 1)  }
{Thünen-Institut, Braunschweig}
{24h}
{Oct 2019  }

\cvhonor
{Essential basics of statistics  }
{Universität Rostock}
{16h}
{Sep 2019  }

\cvhonor
{Gemischte Modelle in R  }
{Thünen-Institut, Braunschweig}
{24h}
{Nov 2018  }

\cvhonor
{Implementation of yield stability assessment with ASReml‑R  }
{Bangladesh Rice Res. Inst., Gazipur}
{4h}
{May 2018  }

\cvhonor
{Statistical analysis with SAS (monthly)  }
{Universität Hohenheim, Stuttgart}
{18h}
{2016‑2018  }

\cvhonor
{Statistical analysis with R (monthly)  }
{Universität Hohenheim, Stuttgart}
{18h}
{2016‑2018  }\end{cvhonors}

\hypertarget{publikationen}{%
\section{Publikationen}\label{publikationen}}

\footnotesize

\hypertarget{bibliography}{}
\leavevmode\vadjust pre{\hypertarget{ref-Friedrichs.2021}{}}%
\CSLLeftMargin{1. }%
\CSLRightInline{Friedrichs, P., Schmidt, P., \& Schmidt, K. (2021).
\emph{Protanopie und protanomalie bei berufskraftfahrern und
berufskraftfahrerinnen - prävalenz und unfallrisiko: = protanopia and
protanomaly among professional drivers: Prevalence and accident risk:
Vols. Heft 319}.
\url{https://bast.opus.hbz-nrw.de/frontdoor/index/index/searchtype/series/id/5/start/1/rows/25/docId/2574}}

\leavevmode\vadjust pre{\hypertarget{ref-Schmidt.2021}{}}%
\CSLLeftMargin{2. }%
\CSLRightInline{Schmidt, K., Friedrichs, P., Cornelsen, H. C., Schmidt,
P., \& Tischer, T. (2021). \emph{Musculoskeletal disorders among
children and young people: Prevalence, risk factors, preventive
measures: A scoping review}. \url{https://doi.org/10.2802/511243}}

\leavevmode\vadjust pre{\hypertarget{ref-Buntaran.2020}{}}%
\CSLLeftMargin{3. }%
\CSLRightInline{Buntaran, H., Piepho, H.-P., Schmidt, P., Rydén, J.,
Halling, M., \& Forkman, J. (2020). Cross--validation of stagewise
mixed--model analysis of swedish variety trials with winter wheat and
spring barley. \emph{Crop Science}, \emph{60}(5), 2221--2240.
\url{https://doi.org/10.1002/csc2.20177}}

\leavevmode\vadjust pre{\hypertarget{ref-Kukowski.2020}{}}%
\CSLLeftMargin{4. }%
\CSLRightInline{Kukowski, S., Schmidt, P., Piepho, H.-P., Röhl, M.,
Hauffe, H.-K., \& Streck, T. (2020). Auswirkungen atmosphärischer
stickstoffeinträge auf magere flachland-mähwiesen in baden-württemberg.
\emph{Natur Und Landschaft}, \emph{95}(2), 58--67.
\url{https://doi.org/10.17433/2.2020.50153773.58-67}}

\leavevmode\vadjust pre{\hypertarget{ref-Schmidt.2019}{}}%
\CSLLeftMargin{5. }%
\CSLRightInline{Schmidt, P. (2019). \emph{Estimating heritability in
plant breeding programs}.
\url{http://opus.uni-hohenheim.de/volltexte/2020/1720/}}

\leavevmode\vadjust pre{\hypertarget{ref-Schmidt.2019b}{}}%
\CSLLeftMargin{6. }%
\CSLRightInline{Schmidt, P., Hartung, J., Bennewitz, J., \& Piepho,
H.-P. (2019). Heritability in plant breeding on a genotype-difference
basis. \emph{Genetics}, \emph{212}(4), 991--1008.
\url{https://doi.org/10.1534/genetics.119.302134}}

\leavevmode\vadjust pre{\hypertarget{ref-Schmidt.2019c}{}}%
\CSLLeftMargin{7. }%
\CSLRightInline{Schmidt, P., Hartung, J., Rath, J., \& Piepho, H.-P.
(2019). Estimating broad-sense heritability with unbalanced data from
agricultural cultivar trials. \emph{Crop Science}, \emph{59}(2),
525--536. \url{https://doi.org/10.2135/cropsci2018.06.0376}}

\leavevmode\vadjust pre{\hypertarget{ref-Schmidt.2018}{}}%
\CSLLeftMargin{8. }%
\CSLRightInline{Schmidt, P., Möhring, J., Koch, R. J., \& Piepho, H.-P.
(2018). More, larger, simpler: How comparable are on-farm and on-station
trials for cultivar evaluation? \emph{Crop Science}, \emph{58}(4),
1508--1518. \url{https://doi.org/10.2135/cropsci2017.09.0555}}

\leavevmode\vadjust pre{\hypertarget{ref-Tulinska.2018}{}}%
\CSLLeftMargin{9. }%
\CSLRightInline{Tulinská, J., Adel-Patient, K., Bernard, H., Líšková,
A., Kuricová, M., Ilavská, S., Horváthová, M., Kebis, A., Rollerová, E.,
Babincová, J., Aláčová, R., Wal, J.-M., Schmidt, K., Schmidtke, J.,
Schmidt, P., Kohl, C., Wilhelm, R., Schiemann, J., \& Steinberg, P.
(2018). Humoral and cellular immune response in wistar han RCC rats fed
two genetically modified maize MON810 varieties for 90 days (EU 7th
framework programme project GRACE). \emph{Archives of Toxicology},
\emph{92}(7), 2385--2399.
\url{https://doi.org/10.1007/s00204-018-2230-z}}

\leavevmode\vadjust pre{\hypertarget{ref-Schmidt.2017}{}}%
\CSLLeftMargin{10. }%
\CSLRightInline{Schmidt, K., Schmidtke, J., Schmidt, P., Kohl, C.,
Wilhelm, R., Schiemann, J., van der Voet, H., \& Steinberg, P. (2017).
Variability of control data and relevance of observed group differences
in five oral toxicity studies with genetically modified maize MON810 in
rats. \emph{Archives of Toxicology}, \emph{91}(4), 1977--2006.
\url{https://doi.org/10.1007/s00204-016-1857-x}}

\leavevmode\vadjust pre{\hypertarget{ref-Zeljenkova.2016}{}}%
\CSLLeftMargin{11. }%
\CSLRightInline{Zeljenková, D., Aláčová, R., Ondrejková, J., Ambrušová,
K., Bartušová, M., Kebis, A., Kovrižnych, J., Rollerová, E., Szabová,
E., Wimmerová, S., Černák, M., Krivošíková, Z., Kuricová, M., Líšková,
A., Spustová, V., Tulinská, J., Levkut, M., Révajová, V., Ševčíková, Z.,
\ldots{} Steinberg, P. (2016). One-year oral toxicity study on a
genetically modified maize MON810 variety in wistar han RCC rats (EU 7th
framework programme project GRACE). \emph{Archives of Toxicology},
\emph{90}(10), 2531--2562.
\url{https://doi.org/10.1007/s00204-016-1798-4}}



\end{document}
